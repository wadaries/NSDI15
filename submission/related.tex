\vspace{-.5em}
\Section{Related work}
\label{sec:related}

\Paragraph{Network verification and synthesis}, despite recent
efforts~\cite{veriflow,NetPlumber,network-verification-in-light-of-pv,dnv},
have not matured into a wanted industry of standard reusable tools
with a general-purpose interface.  General-purpose verification tools,
despite powerful tools like SMT solvers, are not directly applicable
to network size; Domain-specific heuristics are fast but require extra
effort and are not reusable.  Formal synthesis is even harder and
slower: existing tools does not scale
well~\cite{reactive-synthesis-sdn}. By utilizing DB's general query
engine that has been optimized for two decades as the main reasoning
engine, \Sys brings new hope to a networking verifier that strikes a
balance between general-purpose support and realtime performance.  We
would throughly study the power and limits of \Sys's support for
realtime verification and synthesis.

\Paragraph{Views, authorization, and locking} are three inherently
connected concepts in database~\cite{Views-Authorization-and-Locking}:
views expose data, authorization prescribes access to data, and
locking implements authorization. Thus, the view-centric \Sys lends
itself to a rich body of locking based authorization mechanism that is
particularly straightforward and flexible: one may associate lock with
different operations \eg write or/and read, and data of different
granularity \eg entire table, one record, one column, or arbitrary
region defined by a condition.  As such, we envision that \Sys may
hold the solution to the increasingly complicated network security and
privacy requirement (\eg isolation) where users could conveniently
project the network data into views, with authorization granted via
locks.

\Paragraph{Network OS and SDN programming APIs} 
introduce a global network abstraction of the distributed forwarding
plane along with programming APIs that offers reusable
primitives~\cite{onix,nox,composing,sdn-lang-frenetic}. Similar to
pre-database era, where online commercial data management used file
system and general-purpose programming language, we view OS/PL powered
SDN as a preliminary stage of
\Sys~\cite{Date:1971:file-and-data-independence,Cardelli:1985:PL-data-abstraction},
which is restrictive and requires considerable expertise.  \Sys seeks
a more flexible and accessible user interface that is customizable and
managed through simple databases operators that are processed as
transactions.

\Paragraph{Declarative networking}~\cite{declarative-networking,p2}
draws on the natural connection of recursive Datalog (declarative
deductive database language) and network properties such as
reachability, presents a Datalog based platform for specifying
distributed networking services. Later work on declarative network
management~\cite{practical-dn} extends Datalog to policy management for
enterprise networks. Compared to these efforts, \Sys applies DB
techniques in a broader sense, covering the two main DB pillars: data
independence and transaction processing. On the other hand,
declarative networking sheds light on many issues \Sys is facing,
ranging from the details in connecting DB to a real
network~\cite{practical-dn,declarative-networking}, to the
data-independence principle discussed
in~\cite{data-independence-network}\footnote{\cite{data-independence-network}
  deals solely with physical data-independence, whereas logical
  data-independence also plays a key role in \Sys}.

% \Paragraph{Network virtualization} were among the most noticeable
% efforts towards a more manageable network with rich and new
% functionalities before the SDN era. VLAN, VPN, overlay networks,
% infrastructure as a service (IaaS) ... for legacy networks. In
% particular, \textit{recursive virtualization} ... higher layer
% abstraction independent of forwarding plane (lower-layer) changes.
% ...  MPLS, active networks ... enables modification of lower-layer
% that reflects high-level network abstraction updates ...
