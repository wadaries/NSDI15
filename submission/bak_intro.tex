\section{Introduction}
\label{sec:intro}

To bring about the needed abstraction capable of reconciling what a
human mind perfects and how today's network performs, we champion a
shift from OS/PL to database system (DB)
technique~\cite{Abiteboul:1995:alice,db-concept,db-meta}, like it has
taken place for online commercial data management in 1980s. OS/PL was
proven to fall short for mediating between an increasingly complex
online data management system and multiple users of little computer
specialty, whereas DB arose with two abstractions: data independence
and transaction processing. Data independence is the data abstraction
that simplifies human interaction by hiding data storage and
representation, whereas transaction processing abstraction achieves
concurrency and recovery control without severely hurting
performance. By adapting data independence and transaction processing
into networking, we propose a Software-defined network (SDN) database
design that features two-levels abstraction, which we call \TI and \TR
respectively.



To simplify and renovate networking, many recent efforts decouple
logical abstractions from physical infrastructure via operating system
(OS) and programming language (PL)
means~\cite{ethane-sigcomm07,rethinking-enterprise,shenker-tue}.  For
example, software-defined networks (SDN), by exploiting the idea of
network operating system~\cite{onix,nox}, exposes and maintains a
separate network-wide abstraction -- a network representation
(analogous to RIB) and control primitive (routing, access control
\etc).  From a user perspective, the abstraction forms a programming
API~\cite{composing,sdn-lang-frenetic} -- high-level data-structures
and operations that manipulate them.  Similarly, virtualized networks
(VN) separates logical services / functionality from the underlying
physical networks, gives tenants an abstract network representation
(virtual topology, address \etc), through APIs over a spectrum of
granularity.
% This is analogous to pre\hyp{}database day computerized management
% of commercial data, when programmers write applications over
% data\footnote{Unlike the file-system used in conventional operating
% system, today's SDN OS employs a graph databases which is
% essentially a key\hyp{}value storage optimized for graphs.}
% supported by an operating system: For example, \todo the OS provides
% APIs that enable the applications to read and write the appropriate
% data.


% To simplify network operation and ease innovation, the networking
% community sees an endlessly growing new techniques that are built
% around operation system (OS) and programming language (PL) techniques.
% For example, active networks, programmable networks, network
% virtualization, routing control platforms, and software-defined
% networks (SDN)~\cite{shenker-tue}. \todo Despite these OS/PL powered
% efforts, network operation remains manual, tedious, and error-prone.

However, network operation and innovation remain manual and difficult,
due to the grand challenge in finding the right abstractions that
strikes a balance between two conflicting goals: human convenience and
reasonable (feature-rich) system performance, over a distributed,
shared, unreliable network of heterogeneous devices. %  As shown in
% Figure~\ref{fig:abstraction-space},
Existing approaches either achieves simplicity, performance, or
moderate gain on both both.  On one extreme, L2 LAN abstraction,
though simple and configuration-free, does not scale beyond certain
topology (\eg spanning tree) and network size (due to flat address);
On the other, L3 IP-routing offers an end-to-end abstraction that
scales to the entire Interest, however, imposes administration burden
through manual configurations. Abstraction sitting in the middle
includes SDN and virtual networks. SDN reduces program logic
complexity by introducing a simpler logically centralized controller,
at the expense of forcing the programmer to explicitly deal with data
redudency, isolation, consistency, and recovery to get reasonable
performance, a task that is notoriously challenging.  Virtualized
networks, on the other, offers a spectrum of abstractions: the
logically simpler ``one-big-switch'', though gentler to the logical
tenants, requires some efforts when it comes to the mapping into
physical infrastructure; a fuller virtualized abstraction that
supports arbitrary topology, on the other end of the spectrum, can be
realized with a more straightforward mapping, but requires significant
work on the tenant side, actually closer to that needed for a physical
network\footnote{For example, to isolate user programs by different
  tenants, a fuller virtualized network could utilize the tenant
  network namespace as indexes; whereas the the one-big-switch
  abstraction needs to explicitly inspect user program context.}.

% (More examples and details are also in Table~\ref{tab:tradeoff} of
% Appendix~\ref{app:space}.)

% \begin{figure}[h!]
%   \centering
%   \includegraphics[width=.8\linewidth]{figure-source/abstraction-space.pdf}
%   \caption{Abstraction space for networking, divided by human
%     convenience and system performance: existing approaches achieve
%     either, or moderate on both; SDN databases combines the strength
%     of both.}
%   \label{fig:abstraction-space}
% \end{figure}


\TI offers a programmable user-level abstraction exposing a human
interface that simplifies user logic, together with a language that
creates the abstraction on demand, and a mechanism that pushes
abstract operation back into the network device.  The key is that,
rather than seeking for one pre-defined abstraction \footnote{We use
  user program to refer to any control program, \eg network-wide
  controller program by an administrator, or application programs by
  end users with limited network access.}  that fits every existing
user application (and potentially future ones) right, we turn SDN
network (collection of FIBs) into a distributed relational database,
using the SQL language to create user abstraction on demand --
permitting application-specific views that hide irrelevant details as
well as data restructuring. After the creation of the view, more
importantly, the user program shall be able to manipulate the network
through the views. To this end, we utilize DB view mechanisms, namely
view maintenance that keeps abstractions fresh, verifying user
intention against the latest network states (FIB change); and view
update that compiles an abstract view operation into a collection of
network FIBs operations, synthesizing the actual FIB implementation.

\TR, on the other hand, provides a separate transactional device-level
abstraction, is responsible for efficient execution of user programs
that preserves an isolated, sequential, and recoverable semantics. The
key is that, rather than adding concurrency control to user program's
data -- the virtual view abstraction, \TR chooses the switch-level
FIBs abstraction for implemention, hence achieves concurrency and
recovery control without severely hindering system performance. For
example, if transaction processing is implemented by two-phase
locking, rather than grabbing and releasing global locks over the
views, which is in-efficient and complicated, \TR translates the view
locking into FIB lockings that can be directly implemented on relevant
switch hardware. Note that the FIBs correspond to the base tables from
which a user view is derived, therefore, the same SQL query that
creates the view is used to compile the global view locks into the
local switch-level locks.

The rest of the paper is organized as
follows. Section~\ref{sec:design} outlines the SDN database design,
featureing \TI and \TR.  Section~\ref{sec:details} describes \TI in
more details, presents early stage implementation that evaluates the
\TI design. Section~\ref{sec:discussion} discusses the opportunities,
challenges, and limits of SDN database. Section~\ref{sec:related}
connects SDN database to relevant OS/PL
efforts. Section~\ref{sec:conclusion} concludes the paper.

