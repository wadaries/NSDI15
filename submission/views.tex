\Section{View Interface Design}
\label{sec:details}

\subsection{View interface design}

The design goal of \Sys view interface is to combine the strength of
the following:
\begin{itemize}
\item A concise interface that holds only the relevant information,
  structured in a format that eases control logic. The challenge is to
  identify a small set of views that are also expressive enough for
  common networking tasks. (\S~\ref{subsec:view-library})
\item Composition. (\S~\ref{subsec:compose})
\item Built-in services of real-time verification and
  synthesis. (\S~\ref{sec:veri-syn})
\item Performance. The challenge is that a naive implementation of
  relational queries does not scales well for the networking setting
  of SDN, where most interesting abstractions, by nature, will call
  for path-related computation that is recursive, and hence expensive
  in the naive implementation. (\S~\ref{sec:eval})
\end{itemize}

To achieve all the above, \Sys features a library of view primitives.
... create abstractions on the fly that can be categorized in two
groups: (1) the per-flow views ... We also call this type ``local
views'', because ... (2) the network-wide views ... we call this type
``global views''.  (1) enables real-time verification and synthesis;
(2) provides the interface for integrating network-wide service such
as traffic engineering.  This section presents (1,2) in
details. Services and performance are discussed in
\S~\ref{sec:veri-syn} and \S~\ref{sec:eval}.


\subsection{The primitives}
\label{subsec:view-library}

\Paragraph{Per-flow forwarding graph and reachability views}

\begin{sql}
def generate_forwarding_graph (cursor, flow_id):

    fg_view_name = "fg_" + str (flow_id)

    try:
        cursor.execute("""
        CREATE OR REPLACE view """ + fg_view_name + """ AS (
        SELECT 1 as id,
               switch_id as source,
	       next_id as target,
	       1.0::float8 as cost
        FROM configuration
        WHERE flow_id = \%s
        );
        """, ([flow_id]))

    except psycopg2.DatabaseError, e:
        print "Unable to create fg_view table for flow " + str (flow_id)
        print 'Error \%s' \% e    
\end{sql}

\begin{sql}
def generate_reachability_perflow (cursor, flow_id):

    fg_view_name = "fg_" + str (flow_id)
    reach_view_name = "reachability_" + str (flow_id)

    try:
        cursor.execute("""
        CREATE OR REPLACE view """ + reach_view_name + """ AS (
          WITH ingress_egress AS (
               SELECT DISTINCT f1.source, f2.target
               FROM """ + fg_view_name + """ f1, """ + fg_view_name + """ f2
               WHERE f1.source != f2.target AND
                     f1.source NOT IN (SELECT DISTINCT target FROM """ + fg_view_name +""") AND
                     f2.target NOT IN (SELECT DISTINCT source FROM """ + fg_view_name +""" )
               ORDER by f1.source, f2.target),
               ingress_egress_reachability AS (
               SELECT source, target,
                      (SELECT count(*)
                       FROM pgr_dijkstra('SELECT * FROM """ + fg_view_name + """',
                                         source, target, TRUE, FALSE)) AS hops
               FROM ingress_egress)
          SELECT * FROM ingress_egress_reachability WHERE hops != 0
        );""")

    except psycopg2.DatabaseError, e:
        print "Unable to create reachability table for flow " + str (flow_id)
        print 'Error \%s' \% e      
\end{sql}

% This section presents a prototype design of \TI in more details,
% showing preliminary evaluation with promising results.

% \Paragraph{Update views}



\subsection{Composition}
\label{subsec:compose}


