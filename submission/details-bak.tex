\section{Design of SDN Database}

\begin{figure*}[ht!]
        \centering
        \begin{subfigure}[t]{.6\linewidth}\centering
          \includegraphics[width=1\textwidth]{fig-SDNandDB-cropped.pdf}
          % \setlength{\abovecaptionskip}{0cm}
          % \vspace{-2.8em}
          \caption{\textit{\textbf{SDN architecture (left)}},
            \textit{\textbf{database architecture
                (right)}}. \textit{\textbf{SDN networks}} offers a
            centralized logical view of the network infrastructure:
            app/users see the same complex network states, controller
            carefully handles name/device binding;
            \textit{\textbf{database}} bridges user and data by two
            separate data abstractions: logical base tables hide
            physical heterogeneity, external views interface end
            users, \textbf{view maintenance and update} synchronize
            the two.}
          \label{fig:SDNandDB}
        \end{subfigure}
        \hfill
        \begin{subfigure}[t]{.39\linewidth}\centering
          \includegraphics[width=1\textwidth]{fig-SDNasDB-cropped.pdf}
          % \vspace{-1.5em}
          \caption{\textit{\textbf{SDN database}} mediates between the
            users and network infrastructure: \textbf{\textit{logical
                level}} hides device specifics;
            \textit{\textbf{application views}} permit user
            perspective; \textit{\textbf{view maintenance}} keeps user
            logic consistent with network state; \textit{\textbf{view
                update}} compiles user logic into devise\hyp{}wise
            implementation.}
          % \vspace{-1em}
          \label{fig:SDNasDB}
        \end{subfigure}
        \caption{\textbf{Topological\hyp{}independent networking}
          extends data-independence principle into
          networking}\label{fig:overview}
\end{figure*}

The following outlined the key features, opportunities and challenges
of SDN database design: topological\hyp{}independent networking that
simplifies SDN programming via data\hyp{}independence, transactional
networking that frees programmer from concurrency and recovery control
via transaction processing, and finally inter\hyp{}operational
networking that show\hyp{}cases networking innovations that connects
SDN networks via more advanced and recent database techniques.

\Paragraph{Topological\hyp{}independent networking}

The first requirement of very large system management is the ability
of decoupling what a user need from how it is implemented. This is
particularly true in networking: the network configurations and
programs are implemented and enforced device-wise, whereas a network
programmer conceives policy and logic better at a higher level.
% usually topology-wise that constitutes a path (paths).

The coupling of and tension between how networks are organized and
what a human mind perfects result in a formidable programming task,
the complexity of which largely stems from the heterogeneous
infrastructure with a specific topology, rather than from the control
logic itself. From a programmer's perspective, to achieve any
high-level service, he faces a dilemma of either utilizing
topological-independent primitives in a restricted setting, or
achieving a universal solution with extra effort. For example, L2 is
widely adopted for its simplicity regardless of topological specifics,
such as device heterogeneity, topological placement; On the contrary,
IP-routing, which does provide a topological-independent end-to-end
connection, is restricted to L3 and above, requiring manual
configuration.

To this end, SDN databases adopts databases notion of
\textbf{data-independence}, which we call
\emph{topological-independent networking}
(Figure~\ref{fig:SDNasDB}). Topological\hyp{}independence exposes an
abstract view of user's data tailored to the user's decision making
need, provides two functionalities: hiding irrelevant information, and
more importantly, restructuring the exposed data to simplify user
logic, as formulated in the following conjecture:

\Quote{SDN database programs operate on high-level
  application-specific data items, unaware of topological specifics,
  such as individual device heterogeneity and device placement.}

To achieve this, topological\hyp{} provides two separate levels of
data abstractions (Figure~\ref{fig:SDNandDB} (right)) and the tools
that synchronize the two. Logical representation of the heterogeneous
data sources are stored in base tables, whereas user\hyp{}defined
views that permits different application perspectives are derived
tables from the bases on demand.  To synchronize them, view
maintenance mechanism keeps the user\hyp{}defined views consistent
with base tables changes, whereas view update mechanism populates
modification to user views back into the bases
(Figure~\ref{fig:SDNasDB}).  The \textit{separation} of the logical
base tables and the programmable views is key to
topological\hyp{}dependency: while the logical base is still close
enough to the physical device, thus providing reasonable performance;
the programmable views interface the user for human convenience,
drastically simplifying user logic.

% More details are outlined in Section~\ref{sec:details}.

% Representation form the external interface. SDN database
% automatically synchronizes operations on the high-level names and
% changes to the actual devices.  The base table logical
% representation and physical storage details of data, whereas view
% abstraction separates the data abstraction over of which programs
% are developed. ...  ... onto per-device flow-tables are base
% database tables; and the per-path policies are virtual tables (\ie
% database views) derived from the base tables by running queries.
% While the following sections outlines

\Paragraph{Transactional networking}

The second requirement of very large system management is the ability
of permitting concurrent operations in a shared unreliable
environment. 

Transactional networking hides the network reality of a concurrent and
un-reliable execution environment by adapting database transaction
processing into SDN setting.

\Quote{In SDN database, network programs ..., unaware of interleaving
  with .}

While topological\hyp{}independent networking is achieved by mapping
DB data itmes into networking entities, transactional networking is
about porting into SDN the\textit{Read/Write} operations and the
theory of resolving conflicts among them.


\todo

% To achieve reasonable performance, database offers transaction
% processing in a spectrum ranging from expensive strong consistency to
% lightweight eventual consistency.

% \rephrase

Transaction semantics~\cite{Bernstein:concurrency-recovery}: ACID

\noindent\textit{Naive solution} \textit{transaction and views} operation over virtual
views?\\

\noindent\textit{Challenges} Reading operation, traffic



\Paragraph{Integrable/inter\hyp{}operational networking}
\label{ssec:integration-integration}

Having shown the two pillars of SDN networking in one single
administration domain, namely topological\hyp{}independence and
transaction processing. This section outlines how to inter-connect
networks with more advanced and recent DB techniques.

\todo

Example problems: Software defined Internet exchange, integrate
networks of heterogeneous sources into a single consistent one

Naive solutions: (1) federated DB~\cite{Sheth:1990:federated-DB}, (2)
View integration and (3) Semi\hyp{}structured data
query~\cite{McHugh:1997:Lore-DB,Abiteboul97queryingsemi-structured,Buneman:semi-structured}



% \noindent\textit{Challenges}

% \begin{figure*}[htp]
% \centering
% \begin{minipage}[t]{.62\linewidth}
%   \includegraphics[width=\textwidth]{fig-SDNandDB.pdf}
% \caption{Caption}\label{label-a}
% \end{minipage}\hfill
% \begin{minipage}[t]{.37\textwidth}
%   \includegraphics[width=\textwidth]{fig-SDNasDB.pdf}
% \caption{Caption}\label{label-b}
% \end{minipage}
% \end{figure*}


% \begin{figure*}[ht]
%   \centering
% \includegraphics[width=.95\textwidth]{overview.pdf}
% \label{fig:overview}
% \caption{; \textit{SDN databases (right)} simplifies the
%   interaction of app/user and network infrastructure: \textit{views}
%   are programmable and created for specific user/app as desired to
%   expose only the relevant network state in a proper abstract form,
%   \textit{view maintenance} notifies app/user network state chagne in
%   realtime, \textit{view update} compiles app/user program into
%   devise-level configurations}
% \end{figure*}
